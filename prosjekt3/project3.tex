\documentclass[a4paper,english, 10pt, twoside]{article}
\usepackage[utf8]{inputenc}
\usepackage[T1]{fontenc}
\usepackage[english]{babel}
\usepackage{epsfig}
\usepackage{graphicx}
\usepackage{amsfonts, amssymb, amsmath}
\usepackage{listings}
\usepackage{float}
\usepackage[top=2cm, bottom=2cm, left=2cm, right=2cm]{geometry}

%opening
\title{Project 3, FYS4150}
\author{Fredrik E Pettersen\\ fredriep@student.matnat.uio.no}


\begin{document}

\maketitle

\begin{abstract}

\end{abstract}

\section*{About the problem}
The task of this project is to compute, with increasing degree of cleverness, the six dimensional integral used to determine the 
ground state correlation energy between two electrons in a helium atom. We will start off with ``brute force'' Gauss Legendre 
quadrature, proceed to Gauss Laguerre quadrature, and finish off with Monte Carlo integration. We assume that the wave function 
of each electron can be modelled like the single-particle wave function of an electron in the hydrogen atom. The single-particle 
wave function for an electron i in the 1s state is given in terms of a dimensionless variable (we ommit normalization of the wave 
functions)
$$
\mathbf{r}_i = x_i\mathbf{e}_x + y_i\mathbf{e}_y + z_i\mathbf{e}_z
$$
as
\begin{align*}
 \psi_{l,s}(\mathbf{r}_i) = e^{-\alpha r_i}
\end{align*}
where $\alpha$ is a parameter and 
$$
r_i = \sqrt{x_i^2 + y_i^2 + z_i^2}
$$
In this project we will fix $\alpha = 2$ which should correspond to the charge of the Helium atom $Z = 2$.
The ansatz for the two-electron wave function is then given by the product of two one-electron wave funtions.
$$
\Psi(\mathbf{r}_1,\mathbf{r_2}) = \psi(\mathbf{r}_1)\psi(\mathbf{r_2}) = e^{-2\alpha(r_1+r_2)}
$$
The integral we need to solve is the quantum mechanical expectation value of the
correlation energy between two electrons which repel each other via the classical Coulomb
interaction, namely
$$
\langle\frac{1}{\left|\mathbf{r}_1-\mathbf{r}_2\right|}\rangle = \int^\infty_{-\infty}\int^\infty_{-\infty}
\frac{ e^{-\alpha r_i}}{\left|\mathbf{r}_1-\mathbf{r}_2\right|}d\mathbf{r}_1d\mathbf{r}_2
$$

\section*{The algorithm}
The principle algorithms of this project is Gaussian quadrature and Monte Carlo integration. 
What we are actually doing when we integrate a function by the use of the rectangle, trapezodial or Simpsons rule is to 
approximate the integrand with a Taylor polynomial of degree 0,1 and 2 respectively between the integration points. An obvious 
step forward from here is to approximate the entire function with a Taylor polynomial of degree N-1 for N integration points, 
but then we realize that Taylor polynomials are a bit crude. A larger step forward can be obtained by approximating the 
integrand with an orthogonal polynomial, such as Legendre polynomials.  
We begin with the approximation
\begin{align*}
I = \int\limits^b_a f(x)dx = \int\limits_a^b W(x)g(x)dx \simeq \sum\limits_{i=1}^N w_i g(x_i)
\end{align*}
Where $w_i$ are the integration weights.
This is called a Gaussian quadrature formula if it integrates exactly all polynomials $p \in P_{2N-1}$. That is:
$$
\int\limits_a^bW(x)p(x)dx = \sum\limits_{i=1}^Nw_ip(x_i)
$$
Let us now approximate $g(x) \simeq P_{2N-1}(x)$ where $P_{2N-1}(x) = \mathcal{L}_N(x)P_{N-1}(x)+Q_{N-1}(x)$. 
$\mathcal{L}_{N}(x)$ is an orthogonal polynomial e.g. a Lagendre polynomial. We remember that orthogonal polynomials have the 
property
$$
\int\limits_a^b\mathcal{L}_i(x)\mathcal{L}_j(x)dx = A\delta_{i,j}
$$
where $x\in[a,b]$ is determined by the specific polynomial (e.g $x\in[-1,1]$ for Legendre) and A is some orthogonality relation 
also determined by the specific polynomial ($A = \frac{2}{2i+1}$ for Legende). This means that we have the following
$$
\int\limits_a^bf(x)dx \simeq \int\limits_a^bP_{2N-1}(x)dx =\underbrace{\int\limits_a^b\mathcal{L}_N(x)P_{N-1}(x)}_{=0} + 
\int\limits_a^bQ_{N-1}(x)
$$
When we now extrapolate this to our numerics the integrals take the form of sums. To ensure that the term 
$\int\limits_a^b\mathcal{L}_N(x)P_{N-1}(x) = 0$ we evaluate our sums in the points $\mathcal{L}_N(x_i) = 0$. That is
$$
\int\limits_a^bf(x)dx \simeq \sum\limits_{i=1}^{N-1}P_{2N-1}(x_i)w_i = \sum\limits_{i=1}^{N-1}Q_{N-1}(x_i)w_i
$$

The most general way to do the Gaussian Quadrature is to use Legendre polynomials. What we end up having to do in order to 
evaluate our integral is to choose some number of integration points, and then calculate the integration points and weights. 
That is we need to calculate the zeros of a Legendre polynomial of degree N. We will also need to make a mapping from the 
original limits of our integral to the limits of Legendre polynomials, $x\in [-1,1]$. This is done by the function ``gauleg'' 
found in the resources for the course. What this function is first of all to make a mapping from the limits one gives as 
input to the Legendre limits of -1 and 1. \\
   xm = 0.5 * (x2 + x1);\\
   xl = 0.5 * (x2 - x1);\\
After constructing the Legendre polynomial of degree i evaluated at some point x (from the recurrance relation),
the function runs Newtons method to find its zeros starting out with an appriximation\\
  pp = n * (z * p1 - p2)/(z * z - 1.0);\\
  z1 = z;\\
  z  = z1 - p1/pp;\\
and stores this in the vector x after scaling it according to the limits xm and xl. The roots of a Legendre polynomial are 
symmetric, so we actually find two roots for every iteration in ``gauleg''. The integration weights are also calculated through\\
w(i-1) = 2.0 * xl/((1.0 - z * z) * pp * pp);\\
theese are of course also symmetric.\\

Having done the general ``brute force'' gaussian quadrature and gotten very unsatisfying results we notice that the approach of 
first cutting the integral off at $\pm \lambda$ and then mapping theese limits to $\pm 1$ for all the variables 
$x_1,y_1,z_1,x_2,y_2,z_2$ is a rather clumsy one. If we rewrite the integrand into spherical coordinates we can use the Gauss 
Laguerre quadrature for the radial part instead, seeing as this looks like a typical Gauss Laguerre case:
$$
x^{\alpha}e^{-x}
$$
The transformation is
\begin{align*}
\iint \frac{e^{-4(r_1+r_2)}}{\left|\mathbf{r}_1-\mathbf{r}_2\right|}d\mathbf{r}_1\mathbf{r}_2 = 
\idotsint \frac{e^{-4(\sqrt{x_1^2 + y_1^2 + z_1^2}+\sqrt{x_2^2 + y_2^2 + z_2^2})}}{\sqrt{(x_1-x_2)^2+(y_1-y_2)^2+(z_1-z_2)^2}}
dx_1dx_2dy_1dy_2dz_1dz_2\\
 = \idotsint\frac{r_1^2r_2^2\sin(\theta_1)\sin(\theta_2)e^{-4(r_1+r_2)}dr_1dr_2d\phi_1d\phi_2d\theta_1d\theta_2}
 {\sqrt{(r_1cos(\phi_1)sin(\theta_1)-r_2cos(\phi_2)sin(\theta_2))^2
 +(r_1sin(\phi_1)sin(\theta_1)-r_2sin(\phi_2)sin(\theta_2))^2+(r_1cos(\theta_1)-r_2cos(\theta_2))^2}}
\end{align*}
We now look only at the denominator without the square root
\begin{align*}
r_1^2(cos^2(\phi_1)sin^2(\theta_1)+sin^2(\phi_1)sin^2(\theta_1)+cos^2(\theta_1)) + \\
r_2^2(cos^2(\phi_2)sin^2(\theta_2)+sin^2(\phi_2)sin^2(\theta_2)+cos^2(\theta_2)) -\\
2r_1r_2(cos(\phi_1)cos(\phi_2)sin(\theta_1)sin(\theta_2) + sin(\phi_1)sin(\phi_2)sin(\theta_1)sin(\theta_2) + cos(\theta_1)cos(\theta_2)\\
= r_1^2 + r_2^2 -2r_1r_2(cos(\theta_1)cos(\theta_2)+sin(\theta_1)sin(\theta_2)cos(\phi_1 - \phi_2))\\
= r_1^2 + r_2^2 -r_1r_2(cos(\theta_1+\theta_2)(1-cos(\phi_1 - \phi_2))+cos(\theta_1-\theta_2)(1+cos(\phi_1 - \phi_2)))
\end{align*}
So in the end we are left with
\begin{align*}
\int\limits_0^{\infty}\int\limits_0^{\infty}\int\limits_0^{2\pi}\int\limits_0^{2\pi}\int\limits_0^{\pi}\int\limits_0^{\pi}
\frac{r_1^2r_2^2\sin(\theta_1)\sin(\theta_2)e^{-4(r_1+r_2)}dr_1dr_2d\phi_1d\phi_2d\theta_1d\theta_2}
{\sqrt{r_1^2 + r_2^2 -r_1r_2(cos(\theta_1+\theta_2)(1-cos(\phi_1 - \phi_2))+cos(\theta_1-\theta_2)(1+cos(\phi_1 - \phi_2)))}}
\end{align*}
I need to remark that I use this denominator only because I thought I could simplify the expression more than it allready was in 
the project text, and had allready implemented the change before I relised I was wrong. I did not however bother changing it back 
seeing as it was correct, and the chance of introducing typos was rather large.\\
Now, when we intruduce this expression to the Gauss Laguerre quadrature both the $r_1^2r_2^2$ terms and the exponential will be 
baked into the wheight function if we do another small substitution $\rho_1 = 4r_1, \rho_2 = 4r_2$. This substitution will only 
introduce a factor $\frac{1}{1024}$ from inserting $r_i = \frac{\rho_i}{4}$ for all $r_i$ 
$$
\frac{\frac{\rho_1^2\rho_2^2}{4^24^2}sin(\theta_1)sin(\theta_2)e^{-(\rho_1+\rho_2)}\frac{d\rho_1d\rho_2}{4^2}}
{\sqrt{\frac{\rho_1^2}{4^2}+\frac{\rho_2^2}{4^2}-\frac{2\rho_1\rho_2}{4^2}cos(\beta)}} = 
\frac{\frac{1}{4^6}}{\frac{1}{4}}\frac{\rho_1^2\rho_2^2sin(\theta_1)sin(\theta_2)e^{-(\rho_1+\rho_2)}d\rho_1d\rho_2}
{\sqrt{\rho_1^2+\rho_2^2-2\rho_1\rho_2cos(\beta)}}
$$
This clearly results in 
$$
\frac{4}{4^6} = \frac{1}{1024}
$$
Finally we have an expression to send through our six nested for-loops. The Integration points and wheights are now determined 
by the Gauss Laguerre and Gauss Legendre for the radial and angular parts respectively.\\ \\

We do the Monte Carlo simulations in two different ways. The first is a brute force approach where we draw random points for the 
variables $r_1, r_2, \theta_1 , \theta_2 , \phi_1$ and $\phi_2$ in their respective intervals. We then evaluate 
$f(r_1, r_2, \theta_1 , \theta_2 , \phi_1, \phi_2)$ and calculate the mean of $f$ in the area $r_1,r_2 \in [0,\lambda]$, 
$\theta_1,\theta_2 \in [0,\pi]$, $\phi_1,\phi2 \in [0,2\pi]$ and multiply the mean of $f$ with the ``volume'' $V = 4\pi^4\lambda^2$
The reason we evaluate $f$ in spherical coordinates is that we only have to make cutoffs for the upper limits of $r_1$ and $r_2$, 
that is we limit ourselves to $r_1,r_2 \in [0,\lambda]$ when $r_1,r_2 \in [0,\infty)$ is correct. Had we evaluated $f$ in cartesian 
coordinates we would have had to make similar cutoffs for all the variables in both ends.\\
To get a measure of how good the appriximation is we calculate the variance and standard deviation of the result (we neglect the 
covaraiance because it is a heavy computation and because it is assumed to be small).

Next we think a little about what the function $f$ looks like, and realize that we can do importance sampling if we draw numbers 
from the exponential distribution. The integrand is then changed into
$$
\frac{r_1^2r_2^2\sin(\theta_1)\sin(\theta_2)}{\sqrt{r_1^2 + r_2^2 -2r_1r_2\cos(\beta)}}
$$

\section*{Analytic solution}
It is possible to find a closed form solution to the relevant integral, and put shortly it is$\frac{5\pi^2}{16^2}$.\\
With some help from David J Griffiths we can find this value ourselves to verify that it is correct.
We start out with
$$
\int\frac{e^{-4(r_1+r_2)/a}}{\left|\mathbf{r}_1 - \mathbf{r}_2\right|}d\mathbf{r}_1d\mathbf{r}_2
$$
where $a=1$ in our case. We will now fix $\mathbf{r}_1$ so we can do the $\mathbf{r}_2$ integral first. We also orient the 
$\mathbf{r}_2$ coordinate system such that the polar axis lies along $\mathbf{r}_1$. This gives us
$$
\left|\mathbf{r}_1 - \mathbf{r}_2\right| = \sqrt{r_1^2+r_2^2 -2r_1r_2cos(\theta_2)}
$$
which inserted gives
$$
I_{r_2} = \int\frac{e^{-4r_2/a}}{\sqrt{r_1^2+r_2^2 -2r_1r_2cos(\theta_2)}}d\mathbf{r}_2 = 
\int\frac{r_2^2sin(\theta_2)e^{-4r_2/a}}{\sqrt{r_1^2+r_2^2 -2r_1r_2cos(\theta_2)}}dr_2d\theta_2d\phi_2
$$
The integral over $\phi_2$ is trivial and ammounts to $2\pi$. The $\theta_2$ integral is
\begin{align*}
\int\limits_0^\pi\frac{sin(\theta_2)d\theta_2}{\sqrt{r_1^2+r_2^2 -2r_1r_2cos(\theta_2)}} 
= \frac{\sqrt{r_1^2+r_2^2 -2r_1r_2cos(\theta_2)}}{r_1r_2}\Bigg|_0^\pi\\
=\frac{1}{r_1r_2}\left(\sqrt{r_1^2 + r_2^2 +2r_1r_2}-\sqrt{r_1^2+r_2^2 -r_1r_2}\right)\\
=\frac{1}{r_1r_2}[(r_1+r_2)-\lef|r_1-r_2\right|] = 
\begin{cases}
\frac{2}{r_1} \text{, if  } r_2<r_1 \\
\frac{2}{r_2} \text{, if  } r_1<r_2
\end{cases}
\end{align*}

\section*{Results}

\begin{figure}[H]
\centering 
\begin{tabular}{|c|c|c|c|}
\hline
N &Gauss Legendre &Gauss Laguerre & Correct result \\
\hline
10 & 0.0719797 & 0.186457 &0.192765  \\
15 & 0.239088 & 0.189759 & 0.192765  \\
20 & 0.156139 & 0.191081 & 0.192765  \\
25 & 0.195817 & 0.19174 &0.192765  \\
30 & 0.177283 &0.192113& 0.192765  \\
35 & 0.189923 & 0.192343& 0.192765  \\
40 & 0.184417 & 0.192493 &0.192765 \\
45 & 0.189586 &0.192595 &0.192765  \\
50 & 0.18756  & 0.192667 &0.192765 \\
\hline
\end{tabular}
\caption{Results of Gaussian quadrature for increasing N}
\end{figure}


\section*{Stability and precision}
\section*{Final comments}
\section*{Source code}
\end{document}
