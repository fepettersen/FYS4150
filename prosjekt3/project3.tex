\documentclass[a4paper,english, 10pt, twoside]{article}
\usepackage[utf8]{inputenc}
\usepackage[T1]{fontenc}
\usepackage[english]{babel}
\usepackage{epsfig}
\usepackage{graphicx}
\usepackage{amsfonts, amssymb, amsmath}
\usepackage{listings}
\usepackage{float}
\usepackage[top=2cm, bottom=2cm, left=2cm, right=2cm]{geometry}

%opening
\title{Project 3, FYS4150}
\author{Fredrik E Pettersen\\ fredriep@student.matnat.uio.no}


\begin{document}

\maketitle

\begin{abstract}

\end{abstract}

\section*{About the problem}
The task of this project is to compute, with increasing degree of cleverness, the six dimensional integral used to determine the 
ground state correlation energy between two electrons in a helium atom. We will start off with ``brute force'' Gauss Legendre 
quadrature, proceed to Gauss Laguerre quadrature, and finish off with Monte Carlo integration. We assume that the wave function 
of each electron can be modelled like the single-particle wave function of an electron in the hydrogen atom. The single-particle 
wave function for an electron i in the 1s state is given in terms of a dimensionless variable (we ommit normalization of the wave 
functions)
$$
\mathbf{r}_i = x_i\mathbf{e}_x + y_i\mathbf{e}_y + z_i\mathbf{e}_z
$$
as
\begin{align*}
 \psi_{l,s}(\mathbf{r}_i) = e^{-\alpha r_i}
\end{align*}
where $\alpha$ is a parameter and 
$$
r_i = \sqrt{x_i^2 + y_i^2 + z_i^2}
$$
In this project we will fix $\alpha = 2$ which should correspond to the charge of the Helium atom $Z = 2$.
The ansatz for the two-electron wave function is then given by the product of two one-electron wave funtions.
$$
\Psi(\mathbf{r}_1,\mathbf{r_2}) = \psi(\mathbf{r}_1)\psi(\mathbf{r_2}) = e^{-2\alpha(r_1+r_2)}
$$
The integral we need to solve is the quantum mechanical expectation value of the
correlation energy between two electrons which repel each other via the classical Coulomb
interaction, namely

\section*{The algorithm}
\section*{Source code}
\section*{Analytic solution}
\section*{Results}
\begin{tabular}{|c|c|c|c|c|}
\hline
N & $\epsilon_r$ LU decomposition& CPU time LU decomposition &$\epsilon_r$& CPU time tridiagonal decomposition \\
\hline
5 & -12.5 & - & -0.7 & - \\
10 & ? & - & -1.2 & - \\
100 & -1.3 & - & -3.0 & - \\
500 & -1.9 & - & -4.4 & - \\
1000 & -2.2 & 30 & -4.8 & - \\
10000 & -3.2 & 3160 & -5.0 & - \\
$10^5$ & x & out of memory & -5.1 & - \\
$10^6$ & x & out of memory & -5.07 & 20 \\
$10^7$ & x & out of memory & x & 250 \\
$10^8$ & x & out of memory & x & 2380 \\
$1.5*10^8$ & x & out of memory & x & 3480 \\
\hline
\end{tabular} \\ \\
\section*{Stability and precision}
\section*{Final comments}

\end{document}
