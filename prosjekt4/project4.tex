\documentclass[a4paper,english, 10pt, twoside]{article}
\usepackage[utf8]{inputenc}
\usepackage[T1]{fontenc}
\usepackage[english]{babel}
\usepackage{epsfig}
\usepackage{graphicx}
\usepackage{amsfonts, amssymb, amsmath}
\usepackage{listings}
\usepackage{float}
\usepackage[top=2cm, bottom=2cm, left=2cm, right=2cm]{geometry}

%opening
\title{Project 4, FYS4150}
\author{Fredrik E Pettersen\\ fredriep@student.matnat.uio.no}


\begin{document}

\maketitle


\section*{About the problem}
The aim of this project is to simulate the development of a system of spins fixed in a position in the plane. The particles can 
have spin up or down represented by the values $\pm 1$. We simulate using the simplest form of the Ising model in 2D where the energy 
of a particle is given by $E = -J\sum\limits_{\langle kl\rangle}s_{kl}$. The notation on the summation sign indicates a sum over the 
nearest neighbours of the paricle in question. $J$ is here a coupling constant which we will set equal to 1 throughout the project. 
We will also limit ourselves to using only the Metropolis algorithm with periodic boundary conditions meaning that at the boundary 
(say the right boundary) the neighbouring partice to the right of a particle at the boundary is the particle on the left boundary 
with the corresponding coordinates.

\section*{The algorithm}
The algorithm of choise here is the Metropolis algorithm.
\section*{Analytic solution}
In the case where we have a lattice size of $2\times2$ we are able to find a closed form solution for all the important parameters 
in this project. We will start off by finding the energy of the system for all $2^4 = 16$ possible microstates of the system:
begin{table}[H]
 
\begin{table}[H]
\centering
\begin{tabular}{|c|c|c|c|c|c|}
\hline
configuration & multiplicity ($\Omega$)& E & $|M|$ & $M$ & $M^2$\\
\hline
$\begin{matrix}\uparrow \uparrow\\ \uparrow \uparrow\end{matrix}$ & 1 & -8J & 4 & 4 & 16 \\
\hline
$\begin{matrix}\uparrow \downarrow \\ \uparrow \uparrow\end{matrix}$& 4 & 0 & 2 & 2 & 4\\
\hline
$\begin{matrix}\uparrow \uparrow \\ \downarrow \downarrow \end{matrix}$ & 4 & 0 & 0 & 0 & 0 \\
\hline
$\begin{matrix}\uparrow \downarrow  \\ \downarrow \uparrow \end{matrix}$ & 2 & 8J & 0 & 0 & 0 \\
\hline
$\begin{matrix}\uparrow \downarrow \\ \downarrow \downarrow \end{matrix}$ & 4 & 0 & 2  & -2 & 4\\
\hline
$\begin{matrix}\downarrow \downarrow  \\ \downarrow \downarrow \end{matrix}$ & 1& -8J & 4 & -4 & 16 \\

\hline
\end{tabular}
\caption{All the possible spin configurations for a $2\times 2$ system of spins with periodic boudarys.}
\label{table1}
\end{table}
From this we can find the partition function of the $2\times2$ system through the well known formula for the partition function
$$
Z = \sum\limits_E \Omega(E)e^{-\beta E} = 12 + 2e^{-8\beta J} + 2e^{8\beta J} = 4\left(3+\cosh(8\beta J)\right)
$$
We can now find the energy of the system through the relation 
\begin{align*}
 \langle E\rangle = -\frac{\partial \ln\left(Z\right)}{\partial \beta} &= -\frac{1}{4\left(3+\cosh(8\beta J)\right)}\cdot \frac{d}{d\beta}4\left(3+\cosh(8\beta J)\right)\\
 \langle E\rangle &= \frac{-8\beta J \sinh(8\beta J)}{\left(3+\cosh(8\beta J)\right)}
\end{align*}
And by differentiating this expression one more time with respect to $\beta$ we can find the heat capacity
\begin{align*}
 \langle C_V\rangle&= \frac{1}{k_BT^2}\cdot\frac{\partial^2 \ln\left(Z\right)}{\partial^2 \beta} = \frac{-(8J)^2}{k_BT^2}\cdot
 \left(\frac{\cosh(8\beta J)(3+\cosh(8\beta J)) - \sinh^2(8\beta J)}{(3+\cosh(8\beta J))^2}\right)\\
 &= \frac{-(8J)^2}{k_BT^2}\cdot\frac{3\cosh(8\beta J)}{(3+\cosh(8\beta J))^2}
\end{align*}
We can also find the expectation value of the absolute magnetization
\begin{align*}
 \langle |M|\rangle = \frac{1}{Z}\sum\limits_{i=1}^N |M_i|e^{-\beta E_i} = \frac{1}{Z}\left(2\cdot4e^{8\beta J} + 2\cdot (4\cdot 2)\right)
  = \frac{2e^{8\beta J} + 4}{3+\cosh(8\beta J)}
\end{align*}
And finally we can find the magnetic sucepitbility $\chi = \frac{\sigma_M^2}{k_BT} = \frac{\langle M^2\rangle -
\langle M \rangle^2}{k_BT}$. We find the relevant values of $\langle M^2\rangle$ and $\langle M \rangle^2$ from table \ref{table1}.
\begin{align*}
 
\end{align*}

\section*{Results}
\

\section*{Stability and precision}
\section*{Final comments}

\end{document}
